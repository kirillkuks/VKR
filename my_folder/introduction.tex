\chapter*{Введение} % * не проставляет номер
\addcontentsline{toc}{chapter}{Введение} % вносим в содержание

В настоящие время с огромной скоростью растёт интерес людей к компьютерным играм. В частности, наиболее популярны многопользовательские и соревновательные игры. Как следствие, на рынке многопользовательских игр растёт конкуренция, и одним из основных способов завлечения новых игроков являются скины - стилистическое оформление объекта компьютерной игры. В большинстве случаев скины представляют собой статическую текстуру. Несмотря на это они пользуются большой популярностью среди игроков и являются значительным источником дохода для игровых компаний.

Мотивацией к выбору темы послужил заказ, в котором требовалось создать скин, который изображал анимацию похожую на анимацию из интро х/ф <<Матрица>>. В данному случае реализации текстурной анимации путём изменения текстурных координат недостаточно, и требуется альтернативный подход. Также усложняющим условием является возможность наличия разрыва текстурных координат.

Целью работы является разработка эффективного метода, позволяющего создавать разнообразные анимированные текстуры, с учётом изложенных выше трудностей.
