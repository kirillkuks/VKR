%%% Проверка используемого TeX-движка %%%
\usepackage{iftex}[2013/04/04]
\newif\ifxetexorluatex   % определяем новый условный оператор (http://tex.stackexchange.com/a/47579/79756)
\ifXeTeX
    \xetexorluatextrue
\else
    \ifLuaTeX
        \xetexorluatextrue
    \else
        \xetexorluatexfalse
    \fi
\fi

\RequirePackage{etoolbox}[2015/08/02]               % Для продвинутой проверки разных условий

%%% Поля и разметка страницы %%%

\usepackage{pdflscape}                              % Для включения альбомных страниц
\usepackage{geometry}                               % Для последующего задания полей

%%% Математические пакеты %%%
\usepackage{amsfonts,amsmath,amssymb,amscd,amsthm}  % Математические дополнения от AMS
% %amsthm should be loaded after amsmath!!

\usepackage{mathtools}                              % Добавляет окружение multlined

%%%% Установки для размера шрифта 14 pt %%%%
%% Формирование переменных и констант для сравнения (один раз для всех подключаемых файлов)%%
%% должно располагаться до вызова пакета fontspec или polyglossia, потому что они сбивают его работу
\newlength{\curtextsize}
\newlength{\bigtextsize}
\setlength{\bigtextsize}{13.9pt}

\makeatletter
%\show\f@size                                       % неплохо для отслеживания, но вызывает стопорение процесса, если документ компилируется без команды  -interaction=nonstopmode 
\setlength{\curtextsize}{\f@size pt}
\makeatother

%%% Кодировки и шрифты %%%
\ifxetexorluatex
    \usepackage{polyglossia}[2014/05/21]            % Поддержка многоязычности (fontspec подгружается автоматически)
\else
    \RequirePDFTeX                                  % tests for PDFTEX use and throws an error if a different engine is being used
   %%% Решение проблемы копирования текста в буфер кракозябрами
%    \input glyphtounicode.tex
%    \input glyphtounicode-cmr.tex %from pdfx package
%    \pdfgentounicode=1
    \usepackage{cmap}                               % Улучшенный поиск русских слов в полученном pdf-файле
    \defaulthyphenchar=127                          % Если стоит до fontenc, то переносы не впишутся в выделяемый текст при копировании его в буфер обмена
    
%    \usepackage[T2A]{fontenc}                       % Поддержка русских букв
    \usepackage[T2A,T1]{fontenc}
    \usepackage[utf8]{inputenc}[2014/04/30]         % Кодировка utf8
    \usepackage[english, russian]{babel}[2014/03/24]% Языки: русский, английский
\fi
\usepackage{tempora} %TemporaLGCUni of Times type
\usepackage{newtxmath} %math font of Times type
% need to set the monospace=typewritter font
%https://tex.stackexchange.com/questions/213835/using-many-typewriter-fonts-in-a-single-document

\makeatletter %load fonts for cmtt
\providecommand{\EC@ttfamily}[5]{%
	\DeclareFontShape{#1}{#2}{#3}{#4}{
		<-8.5>#50800
		<8.5-9.5>#50900
		<9.5-10.5>#51000
		<10.5-11.5>#51095
		<11.5-13>#51200
		<13-15.5>#51440
		<15.5-18.5>#51728
		<18.5-22>#52074
		<22-27>#52488
		<27-32>#52986
		<32->#53583}{}}
\DeclareFontFamily{T1}{cmtt}{}
\DeclareFontFamily{T2A}{cmtt}{}
\EC@ttfamily{T1}{cmtt}{m}{n}{ectt}
\EC@ttfamily{T1}{cmtt}{m}{sl}{ecst}
\EC@ttfamily{T1}{cmtt}{m}{it}{ecit}
\EC@ttfamily{T1}{cmtt}{m}{sc}{ectc}
\DeclareFontShape{T1}{cmtt}{bx}{n}%
{<->ssub*cmtt/m/n}{}
\DeclareFontShape{T1}{cmtt}{bx}{it}%
{<->ssub*cmtt/m/it}{}
\EC@ttfamily{T2A}{cmtt}{m}{n}{latt}
\EC@ttfamily{T2A}{cmtt}{m}{sl}{last}
\EC@ttfamily{T2A}{cmtt}{m}{it}{lait}
\EC@ttfamily{T2A}{cmtt}{m}{sc}{latc}
\DeclareFontShape{T2A}{cmtt}{bx}{n}%
{<->ssub*cmtt/m/n}{}
\DeclareFontShape{T2A}{cmtt}{bx}{it}%
{<->ssub*cmtt/m/it}{}
\makeatletter

%\makeatletter %load fonts for cmtt
%\providecommand{\EC@ttfamily}[5]{%
%	\DeclareFontShape{#1}{#2}{#3}{#4}{
%		<-8.5>#50800
%		<8.5-9.5>#50900
%		<9.5-10.5>#51000
%		<10.5-11.5>#51095
%		<11.5-13>#51200
%		<13-15.5>#51440
%		<15.5-18.5>#51728
%		<18.5-22>#52074
%		<22-27>#52488
%		<27-32>#52986
%		<32->#53583}{}}
%\DeclareFontFamily{T2A}{cmtt}{\hyphenchar\font\m@ne}
%\EC@ttfamily{T2A}{cmtt}{m}{n}{latt}
%\EC@ttfamily{T2A}{cmtt}{m}{sl}{last}
%\EC@ttfamily{T2A}{cmtt}{m}{it}{lait}
%\EC@ttfamily{T2A}{cmtt}{m}{sc}{latc}
%\DeclareFontShape{T2A}{cmtt}{bx}{n}%
%{<->ssub*cmtt/m/n}{}
%\DeclareFontShape{T2A}{cmtt}{bx}{it}%
%{<->ssub*cmtt/m/it}{}
%\makeatletter

%\makeatletter
%\input{t1lmtt.fd}
%\@namedef{T1+lmtt}{}
%\makeatother


\renewcommand{\ttdefault}{cmtt}
%\renewcommand{\ttdefault}{lcmtt} %покрупнее
%\usepackage[scaled=.85]{DejaVuSansMono} %слишком похож на рубленый
%\newfont{\wasyten}{wasy10} %название команды для вызова / название шрифта



%Другие шрифты:
% математика
%\usepackage[lite]{mtpro2}
%https://pctex.com/mtpro2.html
% текст        
% https://www.ctan.org/pkg/paratype
%       \usepackage[scaled=0.925]{XCharter}[2017/06/25] % Подключение русифицированных шрифтов XCharter
%\usepackage{pscyr}
%    \IfFileExists{pscyr.sty}{}{}  % Красивые русские шрифты
%\fi

%https://tex.stackexchange.com/questions/8260/what-are-the-various-units-ex-em-in-pt-bp-dd-pc-expressed-in-mm
\usepackage{printlen} %для измерения и вывода параменторов шрифтов, отступов, интервалов

\usepackage{bm} %для жирных начертаний символов

\usepackage{csquotes} %to check quotes

%%% Оформление абзацев %%%
\usepackage{indentfirst}                            % Красная строка

%%% Цвета %%%
%\usepackage[dvipsnames,usenames]{color}
\usepackage{colortbl}
\usepackage[dvipsnames, table, hyperref, cmyk]{xcolor} % Вероятно, более новый вариант, вместо предыдущих двух строк. Конвертация всех цветов в cmyk заложена как удовлетворение возможного требования типографий. Возможно конвертирование и в rgb.

%%% Таблицы %%%
\usepackage{longtable}                              % Длинные таблицы
\usepackage{multirow,makecell}                      % Улучшенное форматирование таблиц:
													% multirow - строки на несколько ячеек, 
												
													% makecell - сесколько строк в ячейке.
													% не работает, если внутри, например, \verb|text| -> \texttt{text}
													% аналоги
%https://tex.stackexchange.com/questions/2441/how-to-add-a-forced-line-break-inside-a-table-cell								
						
													

%%% Общее форматирование
%\usepackage{soul} % используется ulem
\usepackage{soulutf8}                               % Поддержка переносоустойчивых подчёркиваний и зачёркиваний
\usepackage{icomma}                                 % Запятая в десятичных дробях



%%% Предметный указатель  ГОСТ 7.78-99 Index %%%
%c обобщенными рубриками или развернутый
%или указатель терминов (в общем случае - произвольное число указателей)
%подключать до hyperref

%\usepackage{makeidx} %возможно, необходимо подключить И/ИЛИ пройти Tools-> Commands -> MakeIndex

\usepackage{imakeidx} 
%\indexsetup{level=\section*,toclevel=section,noclearpage}
\makeindex[program=makeindex,
options=-s template_settings/common/myindex.ist, %подключаем стилевой файл для форматирования вывода
name=ru, % префикс для русских указателей 
% если убрать <<ru>>, то для работы дефолтового придется вручную включать Tools-> Commands -> MakeIndex
title={\chapterLight{} 
%   \hrule{}
	Предметный указатель
%	\hrule{}
} 
%,columns=1 %по умолчанию 2
]
\makeindex[program=makeindex,
options=-s template_settings/common/myindex.ist, %подключаем стилевой файл для форматирования вывода
name=en, % префикс для английских указателей
title={\chapterLight{}
%	\hrule{}
	Index
%	\hrule{}
} 
%,columns=1 %по умолчанию 2
] 
%убрать добавление <<title>> в содержание:
%\noindexintoc %not to add index title in PURE makeidx %intoc is false by default with imakeidx


%       https://tex.stackexchange.com/a/132415/44348
%\makeatletter
%% we want hyphenation also in the first word
\renewcommand{\@idxitem}{\par\hangindent40\p@\hspace{0pt}\ignorespaces}
%% we don't want a page break before a subitem %implemented in the previous one
%%\renewcommand\subitem{\@idxitem\nobreak\hspace*{20\p@}}
%\makeatother


%%% Фиксация плавающих объектов





%%% Гиперссылки %%%
\usepackage{hyperref}[2012/11/06]

%%% Изображения %%%
\usepackage{graphicx}[2014/04/25]                   % Подключаем пакет работы с графикой

%%% Списки %%%
\usepackage[shortlabels]{enumitem} % shortlabels для того, чтобы изменять токены в списках с дефолтных (иерархическая структура) на произвольныею

%%% Подписи %%%
\usepackage{caption}[2013/05/02]                    % Для управления подписями (рисунков и таблиц) % Может управлять номерами рисунков и таблиц с caption %Иногда может управлять заголовками в списках рисунков и таблиц


\usepackage{subcaption}[2013/02/03]                 % Работа с подрисунками и подобным

%%% Счётчики %%%
%\usepackage[figure,table]{totalcount}               % Счётчик рисунков и таблиц. Взамен используется xassoccnt 
\usepackage{totcount}                               % Пакет создания счётчиков на основе последнего номера подсчитываемого элемента (может требовать дважды компилировать документ)
\usepackage{totpages}                               % Счётчик страниц, совместимый с hyperref (ссылается на номер последней страницы). Желательно ставить последним пакетом в преамбуле

\usepackage{xassoccnt} % для подсчета сумм приложений, рисунков, таблиц 


%%% Продвинутое управление групповыми ссылками (пока только формулами) %%%
\ifxetexorluatex
    \usepackage{cleveref}                           % cleveref корректно считывает язык из настроек polyglossia
\else
    \usepackage[russian]{cleveref}                  % cleveref имеет сложности со считыванием языка из babel. Такое решение русификации вывода выбрано вместо определения в documentclass из опасности что-то лишнее передать во все остальные пакеты, включая библиографию.
\fi
\creflabelformat{equation}{#2#1#3}                  % Формат по умолчанию ставил круглые скобки вокруг каждого номера ссылки, теперь просто номера ссылок без какого-либо дополнительного оформления



\ifnumequal{\value{draft}}{1}{% Черновик
    \usepackage[firstpage]{draftwatermark}
    \SetWatermarkText{DRAFT}
    \SetWatermarkFontSize{14pt}
    \SetWatermarkScale{15}
    \SetWatermarkAngle{45}
}{}

